\section{green program}

{\bf green} program computes for a single event and a given
set of stations, the Green functions. For each station,
a station-channel list specifying the orientation of the sensors must
be input.
The program can optionally compute Green functions (a) for
the Z component alone, (b) for 3-components  with the standard ZNE sensor
orientation (see Appendix C), or (c) for 3-components with the Z component 
directed ''up" and two orthogonal horizontal components oriented 
arbitrarily. To set up station information, it is necessary to create 
a flat file database consisting of the two relations
{\it .site}, and {\it .sitechan} in the CSS 3.0 database format.
It is preferable to create the relations by a program, for example, by
IRIS's {\bf rdseed} program or by selection from a global database.
Other input includes the eigenfunctions which are collected from various databases
containing {\it .eigen} relations for spheroidal or toroidal modes. 
For each sensor, {\bf green} computes and stores in an output {\it .wfdisc} relation
six Green functions, one per moment tensor component. 

\subsection {Command line}

\noindent {\bf green}

\noindent {\bf green} $<$ {\it parameter\_file}

\noindent {\bf green} $<<$ WORD \\
\noindent ........  \\
\noindent ........  \\
\noindent WORD

\noindent There are three different ways to start the {\bf green} program:
\begin{itemize}
\item interactive dialog for setting the input parameters;
\item single shell command with redirection of the standard input to a parameter
file {\it parameter\_file} containing exactly the same information and
in the same order as in the interactive dialog - one answer per line;
\item direct shell script. In this case, the contents of the parameter file are
directly included in the shell script between the delimiters ``WORD". See examples.
\end{itemize}
\noindent The input parameter file consists of six lines:
\begin{description}
\item[Line 1:] {\it in\_dbname } \\
{\it in\_dbname } is the input database name for the {\it .site} and {\it .sitechan} relations. \\
  {\bf Format:} Any string up to 256 characters long.
\item[Line 2:] {\it db\_list} \\
  This is the path to the file defining  the list of database names containing
the eigenfunctions - one
  name per line. 
  Each name refers to the database in which the {\it .eigen} relation resides. \\
  {\bf Format:} Any string up to 256 characters long.
\item[Line 3:] {\it cmt\_event}\\
  This is the path to the file with the CMT solution for a single event. It includes 
  the CMT location, the seismic moment tensor components, the scalar moment, the focal 
  planes, the source half duration time, and the output time step of the synthetic 
  seismograms. \\
  {\bf Format:} Any string up to 256 characters long.
\item[Line 4:] {\it fmin, fmax} \\
  {\it fmin, fmax} define the frequency range to be selected from the input
  eigenfunction databases. All modes with frequences out of this range are 
  rejected. \\
  {\bf Format:} Unformatted.
\item[Line 5:] {\it nsamples} \\
  This is the number of samples in the synthetic seismograms. All synthetic 
  seismograms start from the source time. \\
  {\bf Format:} Unformatted.
\item[Line 6:] {\it out\_dbname} \\
  This is the output database name including only the {\it .wfdisc} relation for 
  unit Green functions. Each row of the {\it .wfdisc} relation refers to the binary 
  file with 6 multiplexed Green functions; i.e., one tensor component. \\
  {\bf Format:} Any string up to 256 characters long.
\end{description}
%
% Input data
%
\subsection {Input data}

\textbf{\large \emph{in\_dbname.}} 
  This database must exist and must include
  {\it .site} and {\it .sitechan} relations. 
  The {\it .sitechan} relation provides the station-channel list used
by the program. Using these relations, the
  {\bf green} program defines for each station 
  a certain number of channel groups. Each group consists of the single 
  Z-component channel or a triple of channels (Z-component, and two horizontal 
  components with sensor directions defined by the {\it hang} field in the
sitechan relation).
  Before grouping, the {\bf green} program sorts the {\it .sitechan} file
by its ({\it sta, chan}) fields. After sorting, it also removes all duplicate 
  rows (equal station and channel code). Note that the channel field,
  {\it chan}, in the {\it .sitechan} relation is a
  three-character channel name following the SEED format convention. 
  The first two characters define the type of channel (e.g., BH, LH, etc.)
unique, for each
  station's group.
  The third character must be the component name: Z (upper case, for vertical) 
  and other symbols (any case, for the horizontals). \\
\\
\textbf{\large \emph{db\_list.}} This file allows 
  joining an unlimited number of {\bf eigcon} outputs for spheroidal, toroidal, or
  radial modes. The single requirement is that intersection between these
files is null. Each desired eigenfunction must be present and unique in
  some {\bf eigcon} output. \\
\\
\textbf{\large \emph{cmt\_event.}} This file consists of a single line with the 
  following fields:  \\
\begin{quote}
{\it evid, year, jday, hour, min, sec, lat, lon,
  depth, step, halfd, }\\
  $M_0$, $M_{rr},\;M_{\theta \theta},\;M_{\varphi\varphi},\;
  M_{r \theta},\; M_{r \varphi},\; M_{\theta\varphi},$ \\
  $M_{n},$ {\it strike1, dip1, slip1, strike2, dip2, slip2,}.
\end{quote}
where, 
\begin{quote}
\begin{description}
\item[{\it evid}] is the event identifier, 8 characters long 
\item[{\it year}] is the event year of form {\bf yyyy} 
\item[{\it jday}] is the event day starting from the beginning of the year 
\item[{\it hour}] is the event hour
\item[{\it min}] is the event minutes
\item[{\it sec}] is the event seconds with decimal fraction
\item[{\it lat}] is the event geographical latitude, (degree)
\item[{\it lon}] is the event geographical longitude, (degree)
\item[{\it depth}] is the event depth, (km)
\item[{\it step}] is the time step for Green functions and seismograms, sec
\item[{\it halfd}] is the source half time duration, sec
\item[$M_0$] is the scalar tensor moment, (dyn$\cdot$cm)
\item[$M_{rr},$] $M_{\theta \theta},\;M_{\varphi\varphi},\;
M_{r \theta},\; M_{r \varphi},\; M_{\theta\varphi}$ are moment 
  tensor components normalized by the coefficient $M_n$.
\item[$M_n$] is the normalization coefficient for the tensor components,
(dyn$\cdot$cm)
\item[{\it strike1,}] {\it dip1, slip1} are the first fault plane
  solution, (degree)
\item[{\it strike2,}] {\it dip2, slip2} are the second fault plane
  solution, (degree)
\end{description}
\end{quote}

\subsection {Output data}

\textbf{\large \emph{out\_dbname.}} Program {\bf green} creates a {\it .wfdisc}
relation in database {\it out\_dbname} and a  multiplexed binary file contained
the Green function waveforms.
The binary file contains six Green functions in  an order corresponding to the
tensor components:
$M_{rr},\; M_{\theta \theta},\;M_{\varphi\varphi},\;
M_{r \theta},\; M_{r \varphi},\; M_{\theta\varphi}$. The total length,
$nsamp$, of the file is equal to $6*nsamples$. 

\subsection{Messages}
Program {\bf green} prints out on the standard output
device the following messages:

\noindent \textbf {\emph{Dialog messages}}. Copy of input/output dialog
or parameter file in dialog form. For example,

\noindent \texttt {enter path to db with sta \& stachan: \\
\emph{RDSEED\_rdseed} \\
enter name of file within list of nmodes db: \\
\emph{db\_list} \\
enter input CMT file name: \\
\emph{cmt\_event} \\
min and max frequencies to be considered (mHz) : \\
\emph{0.  260.} \\
enter \# pts in greens fns .le.  30000 : \\
\emph{4000} \\
enter Green functions output db file name: \\
\emph{green}}

\noindent \textbf {\emph{Warning, error, and info messages}}.

\noindent 1. \texttt{ ============= Program green ====================}
\begin{quote}
Info message. Shows that program {\bf green} starts.
\end{quote}
\noindent 2. \texttt{WARNING: green: \# of points in Green functions is stripped to nnn}
\begin{quote}
Warning message. The number of requested points for the Green functions
exceeds the maximum allowed value {\tt mseis}. The number is reduced to {\tt mseis}
and the program continues to run. To increase the maximum value, change  {\tt mseis} to
a new bigger value in the {\tt green.h} header and recompile the {\bf green} program.
By default, {\tt mseis=30000}.
\end{quote}
\noindent 3. \texttt{WARNING: green: \# of channels is stripped to 3}
\begin{quote}
Warning message. Number of sensors (channels, components) in a group cannot 
be greater than 3.
Only the three first sensors are taken into account. 
\end{quote}
\noindent 4. \texttt{WARNING: green: \# of channels is stripped to 1}
\begin{quote}
Warning message. Number of sensors in a group cannot be equal to 2.
Only the first sensor is taken into account. 
\end{quote}
\noindent 5. \texttt{WARNING: green: Channel: \# 1 is not vertical. Sequence ignored}
\begin{quote}
Warning message. Channel with Z component must be the first in the group.
The group of channels is rejected.
\end{quote}
\noindent 6. \texttt{WARNING: green: Channel: \# nnn is not horizontal. Sequence ignored}
\begin{quote}
Warning message. Second and third channels in the group must be horizontal
components. nnn is the channel number in the group. The group of channels is rejected.
\end{quote}
\noindent 7. \texttt{ERR010: green: max l = nnn, must be .le. ml}
\begin{quote}
Error message. The max angular order number {\tt nnn} for some mode {\tt n}
exceeds the parameter {\tt ml} set up in the {\tt green.h} header.
Program is terminated.
To avoid the problem, make {\tt ml} bigger than {\tt nnn} in the {\tt green.h} header
and recompile the program. It is not recommended to make {\tt ml} greater than
10000, which leads to inaccuracy in computing the associated Legendre polynomials.
By default, {\tt ml = 6000}.
\end{quote}
\noindent 8. \texttt{ERR011:eigen: flat and bin indices are different}
\begin{quote}
Error message. Broken input {\it .eigen} relation. Program is
terminated. Check or create {\it .eigen} again.
\end{quote}
\noindent 9. \texttt{ERR012: green: \# sph. modes in band exceed max 
allowed number nnn}
\begin{quote}
Error message. The total number of spheroidal modes in the band exceeds the
maximum allowed
number of {\tt nnn}. Program is terminated. Increase parameter {\tt meig} in the
{\tt green.h} header and recompile the program. By default, {\tt meig = 200000}.
\end{quote}
\noindent 10. \texttt{ERR012: green: \# tor. modes in band exceed max allowed number nnn}
\begin{quote}
Error message. The total number of toroidal modes in the band exceeds 
the maximum allowed
number {\tt nnn}. Program is terminated. Increase parameter {\tt meig} in the
{\tt green.h} header and recompile the program. By default, {\tt meig = 200000}.
\end{quote}
\noindent 11. \texttt{green: \# sph. modes in band = nnn  must be .le. meig}
\begin{quote}
Info message. \texttt{nnn} is the total number of spheroidal modes in all
input databases. \texttt{meig} is the maximum allowed number of spheroidal modes.
\end{quote}
\noindent 12. \texttt{green: \# tor. modes in band = nnn  must be .le. meig}
\begin{quote}
Info message. \texttt{nnn} is the total number of toroidal modes in all
input databases. \texttt{meig} is the maximum allowed number of toroidal modes.
\end{quote}
\begin{description}
\item[13.] \texttt{green: evid date\&time lat = $\pm$dd.ddd, lon = $\pm$ddd.ddd}
\item[$\;\;\;\;\;$] \texttt{green:        source depth = ddd.d km}
\item[$\;\;\;\;\;$] \texttt{green: step =    d.ddd sec, nsamples =  ddddd}
\end{description}
\begin{quote}
Info message. This message outputs part of the {\it cmt\_event} file: 
event information, Green function's time step and number of samples. 
See description of the {\it cmt\_event} file in Section 5.2.
\end{quote}
\noindent 14. \texttt{green: Input dbname : xxxxx}
\begin{quote}
Info message. xxxxx is the name of the input database for the {\it .site} and 
{\it .sitechan} relations.
\end{quote}
\noindent 15. \texttt{green: Station: code     lat   lon , Channels: nnn}
\begin{quote}
Info message. This message specifies the station code, station coordinates, 
and number of components nnn for the selected group of sensors.
\end{quote}
\noindent 16. \texttt{green: Channel: \#  nnn  code   chan  hang vang}
\begin{quote}
Info message. This message specifies a separate sensor (channel) for
the group of sensors. nnn is the current number in the group, code is the
station code, chan is the channel name. hang (horizontal angle) and vang 
(vertical angle) are sensor orientation
in space. See {\it .sitechan} relation in Section 8.
\end{quote}
\noindent 17. \texttt{green: Epicentral Distance :    ddd.ddd}
\begin{quote}
Info message. ddd.ddd is the station-event distance in degrees.
\end{quote}
\noindent 18. \texttt{green: Azimuth of Source :    ddd.ddd}
\begin{quote}
Info message. ddd.ddd is the azimuth of a station to the source point.
in degree.
\end{quote}
\noindent 19. \texttt{green: Azimuth of Source :    ddd.ddd}
\begin{quote}
Info message. ddd.ddd is the azimuth of  the station to the source point
 in degrees.
\end{quote}
\noindent 20. \texttt{green:   nnn code    chan  date\&time  step   mmm}
\begin{quote}
Info message. This message informs about some attributes specifying the
Green functions. nnn is the global number of Green functions set, code is the
station name, chan is the channel name, date\&time is the event origin 
time, step is the Green function sampling step in seconds, and mmm is 
the total number of samples for the block of 6 Green functions. 
\end{quote}
