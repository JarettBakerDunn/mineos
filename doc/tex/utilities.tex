\section{The Utilities User's Manual}
\subsection{cucss2sac}
\noindent {\bf NAME}
\vred
\begin{itemize}
\vred
\item[]{\bf cucss2sac} - convert CSS 3.0 waveforms to SAC binary or ASCII files
\vred
\end{itemize}
\vred
\noindent {\bf SYNOPSIS}
\vred
\begin{itemize}
\vred
\item[] {\bf cucss2sac [-a [-n]] db\_name out\_SAC\_dir}
\vred
\end{itemize}
\vred
{\bf OPTIONS}
\vred
\begin{itemize}
\vred
\item[]    {\bf -a} - generate ASCII files
\vred
\item[]    {\bf -n} - suppress header output in ASCII files. Used together with {\bf -a} option.
\vred
\end{itemize}
\vred
{\bf DESCRIPTION}
\vred
\begin{quote}
\vred
{\bf cucss2sac} utility converts a CSS 3.0 {\it .wfdisc} relation into a set 
of SAC binary files or  ASCII files. Output files are stored in 
{\bf out\_SAC\_dir} directory. This program has been designed 
especially for the {\bf Mineos} package due to bugs in the standard IRIS 
{\bf css2sac} utility. {\bf cucss2sac} supports limited capabilities 
and is not a complete substitution of the standard {\bf css2sac} 
utility. For example, {\bf cucss2sac} only supports the CSS3.0 schema 
and two input CSS binary data formats: ``t4" (BIG\_ENDIAN, single 
precision floating point) and ``f4" ( LOW\_ENDIAN, single precision 
floating point). {\bf cucss2sac} converts CSS binary data ``as is" 
without any corrections or modifications. For the {\bf Mineos} package, 
{\bf cucss2sac} may be used for the conversation of Green functions 
or synthetic seismograms to binary SAC or ASCII files.

A CSS database {\bf db\_name} must include at least one file 
db\_name.wfdisc file containing CSS format information about the 
waveforms. It is also desirable to add to the database {\it .origin} and 
{\it .site} relation tables (files db\_name.origin and db\_name.site) 
to get a more complete SAC header.
Note that the {\it .origin} relation keeps information about event location 
and the {\it .site} relation keeps station locations. If {\it .origin} is present 
in the database, {\bf cucss2sac} uses only the first line from this 
file. In the case of ASCII output, all available parameters go to the
ASCII header - three lines in the beginning of the file. 
For example, output file for a synthetic seismogram looks as follows:

\texttt{EVENT:   2000(014)-23:37:10.800 947893030.80000 25.3900 101.4000 33.0000 \\
   STATION: ALE     LHE       82.5033  -62.3500  0.0000 \\
   DATA:    2000(014)-23:37:10.800  947893030.80000  8000 1.000000 \\
     0.0000000E+00   5.7025738E+00 \\
     1.0000000E+00   1.9508668E+00 \\
     2.0000000E+00  -4.1160989E+00 \\
     3.0000000E+00  -4.1094885E+00 \\
          ...... \\
          ...... \\
          ...... \\
     7.9970000E+03   1.5211651E+01 \\
     7.9980000E+03   2.7408613E+01 \\
     7.9990000E+03   3.5795624E+01}

where,
the \texttt{EVENT} line represents event information: origin time 
(human and epoch), latitude (deg), longitude (deg), and depth (km); 
the \texttt{STATION} line represents station information: station code,
channel name, latitude (deg), longitude (deg), and depth = 0; the 
\texttt{DATA} line represents data parameters: waveform 
starting time (human and epoch), number of samples, and sampling step 
(sec). The rest of the lines, the body, is two columns table. The first 
column represents relative time from the beginning of waveform in 
seconds and the second ones waveform samples in nm, nm/s or nm/s$^2$, 
according to the {\bf syndat} output format.

If the presence of the header is not desired, suppress it with option 
{\bf -n} (no header).

The output ASCII file format for the Green functions is different. 
The header is the same, but the body is a 7 column table. The first 
column is relative time, as before, and the other 6 columns are the 6 
unit Green functions ($G_{rr}, G_{\theta\theta}, G_{\varphi\varphi}, 
G_{r\theta}, G_{r\varphi}, G_{\theta\varphi}$) for the chosen 
component - vertical or some horizontal. The order of the components 
is the same as the order of the seismic moment tensor components 
$M_{ij}$ defined in the {\bf syndat} program. So, the corresponding 
synthetic seismogram S (IN ACCELERATION !!!, nm/s$^2$) is given by 
the formula

            \[ S = 10^{-18}\;\sum_{ij} G_{ij}\;M_{ij} \]

where $M_{ij}$ are given in dyne*cm for $ij = rr, \theta\theta,
\varphi\varphi, r\theta, r\varphi, \theta\varphi$.

NOTE: According to SAC complaince, event depth is stored in the SAC 
header in meters.
\end{quote}
\vred
{\bf EXAMPLES}
\vred
\begin{quote}
\vred
\textbf{cucss2sac Syndat Syndat\_SAC  \\
    cucss2sac green  green\_SAC \\
    cucss2sac -a Syndat Syndat\_ASC \\
    cucss2sac -a -n Syndat Syndat\_ASC\_NOHEADER} 
\vred
\end{quote}
\newpage
% EIGEN2ASC
\subsection{eigen2asc}
\noindent {\bf NAME}
\vred
\begin{itemize}
\vred
\item[] {\bf eigen2asc} - convert {\it .eigen} relation table to ASCII files
\vred
\end{itemize}
\vred
\noindent {\bf SYNOPSIS}
\vred
\begin{itemize}
\vred
\item[] {\bf eigen2asc [-n] nmin nmax lmin lmax db\_name out\_dir}
\vred
\end{itemize}
\vred
\noindent {\bf OPTIONS}
\vred
\begin{itemize}
\vred
\item[] {\bf -n} - suppress header output in ASCII files.
\vred
\end{itemize}
\vred
\noindent {\bf DESCRIPTION}
\vred
\begin{quote}
\vred
{\bf eigen2asc} utility converts some part or the whole {\it .eigen} 
relation into a set of ASCII files. Program {\bf eigen2asc} searches 
in the db\_name.eigen file for all eigenfunctions whith mode numbers 
n, l satisfyng the following conditions: {\bf nmin} $\le$ n $\le$ 
{\bf nmax}, {\bf lmin} $\le$ l $\le$ {\bf lmax}, and converts 
eigenfunctions to ASCII files. Output files are stored in 
{\bf out\_dir}. Each output file  consists of the header and the body.
The header is a single line including the first 10 fields of the {\it .eigen}
relation, see  Section 8. The option {\bf -n} excludes the header 
output. The body is a three-column (radial, toroidal modes) or seven 
column (speroidal mode) table. The first body column is radius in 
meters and the others columns contain the eigen functions and their 
derivatives by radius (in the internal, normalized units) - 
U, U', V, V', P, P'   for spheroidal modes and W, W' for other modes. 
For more deails, see Section 3.  The output file name has form of 
{\bf X.nnnnnnn.mmmmmmm.ASC}, where, letter {\bf X} is ``S" for 
spheroidal modes or ``T" for toroidal modes, {\bf nnnnnnn}
is the number n, and {\bf mmmmmmm} is the mode number l.
\end{quote}
\vred
{\bf EXAMPLES}
\vred
\begin{quote}
\vred
{\bf eigen2asc 0 1 2 500 test\_S Eigen\_S\_ASC \\
eigen2asc -n 0 1 2 500 test\_S Eigen\_S\_no\_headers\_ASC }
\end{quote}
% ENDI
\subsection{endi}
\vred
{\bf NAME}
\vred
\begin{itemize}
\vred
\item[] {\bf endi} - in-place file swapping with fixed width
\vred
\end{itemize}
\vred
{\bf SYNOPSIS}
\vred
\begin{itemize}
\vred
\item[] {\bf endi nw file1 [ file2 ... filen]}
\end{itemize}
\vred
{\bf DESCRIPTION}
\vred
\begin{quote}
This utility is used when you transfer binary files of {\it .wfdisc} or
{\it .eigen} relations to a platform with  a different byte order.
The utility changes the bytes order from BIG\_ENDIAN to LOW\_ENDIAN
and vice versa. {\bf endi} sequentionally reads the file from the 
list {\bf file1 [ file2 ... filen] } into memory as an unsigned 
character string , swaps sequential groups of {\bf nw} bytes long 
starting from the beginning of file, and stores the
swapped data into the same place as before (swapping in-place).
If the last group has a length less than {\bf nw} it stays unswapped.
\end{quote}
\vred
{\bf EXAMPLES}
\vred
\begin{quote}
Example 1. Let w.00001 and w.00002 the binary real*4 data files from 
some {\it .wfdisc} relation created on a PC computer. These files have 
internal LOW\_ENDIAN data representation created, for example, by 
a direct access FORTRAN WRITE statement. After applying the command:

  $...>$ {\bf endi} 4 w.00001 w.00002

we change the order of 4 byte words to BIG\_ENDIAN for transfering
these files on a computer with BIG\_ENDIAN byte order, say SUN Ultra
machine,  where we may use a direct access READ statement to read 
data as real*4 variables.
\end{quote}
% SIMPLEDIT
\subsection{simpledit}
\vred
{\bf NAME}
\vred
\begin{itemize}
\vred
\item[] {\bf simpledit} - filter to create {\it .site} and {\it .sitechan} CSS 
relations from an input ASCII file
\vred
\end{itemize}
\vred
{\bf SYNOPSIS}
\vred
\begin{itemize}
\vred
\item[] {\bf simpledit ascii\_file db\_name}
\end{itemize}
\vred
{\bf DESCRIPTION}
\vred
\begin{quote}
\vred
{\bf simpledit} is a simple filter program. It converts a manually created
(text editor) unformatted ASCII file {\bf ascii\_file} with station and
channel information into CSS3.0 {\bf db\_name.site} and 
{\bf db\_name.sitechan} relation tables (files). The prefix {\bf db\_name}, 
coming from the command line, is the database name.
In the case of real data, there is another way to create  these relations. 
Use the IRIS {\bf rdseed} program, which provides  SEED volume conversation
to CSS format. Note, CSS format is strictly defined, so do not
create {\it .site} and {\it .sitechan} tables manually. This leads to numerous
format errors and as a result to crash the {\bf Mineos} programs. 
In case of a lot of station data, a better solution is to write a program,
which creates {\bf ascii\_file} or {\it .site} and {\it .sitechan} files directly.

{\bf \emph{Description of the {\bf ascii\_file} file.}} The {\bf ascii\_file}
file is a set of ASCII text lines. Each line describes a single station or a
single channel. All channel lines, placed after some station line, belong
to that station. 
\newpage
For example, consider the following file: 

\texttt{ANMO 34.9502 -106.4602    1.6890 'Albuquerque, New Mexico, USA' \\
@ LH1  150.0000  280.0   90.0 \\
@ LH2  150.0000   10.0   90.0 \\
@ LHZ   89.3000    0.0    0.0 \\
BJT  40.0183  116.1679    0.1370 'Baijiatuan, Beijing, China' \\
@  LHE   60.0000   90.0   90.0 \\
@  LHN   60.0000    0.0   90.0 \\
@  LHZ   60.0000    0.0    0.0 \\
@  BHE   60.0000   90.0   90.0 \\
@  BHN   60.0000    0.0   90.0 \\
@  BHZ   60.0000    0.0    180.0 \\
BILL  68.0651  166.4524    0.2990 'Bilibino, Russia' \\
@       LHZ   0.0000    0.0    0.0 \\
CCM  38.0557  -91.2446    0.1710 'Cathedral Cave, Missouri, USA' \\
@   LHE   51.0000   90.0   90.0 \\
@   LHN   51.0000    0.0   90.0 \\
@   LHZ   51.0000    0.0    0.0 }
\end{quote}
Each line in the file consists of some number of fields separated with
one or more space characters. If a field has an internal space 
character, like in station full name, it must be surrounded with  
single quotes.  The first field starts with the first position in the 
line. A station line has the fields:
{\it sta, lat, lon, elev, staname}. See description in Section 8, {\it .site}
relation table. A channel line starts with text field ``@". The other
fields are: {\it chan, edepth, hang, vang}. See description in Section 
8, {\it .sitechan} relation. The standard orientation of a sensor ''XX"
is defined as
\begin{quote}
\texttt{@   XXE   dd.dddd   90.0   90.0 \\
@   XXN   dd.dddd    0.0   90.0 \\
@   XXZ   dd.dddd    0.0    0.0 }
\end{quote}
where, \texttt{dd.dddd} is a sensor {\it edepth}.
Note that the order of channels for the given station is not important.
\newpage
% CREATE_ORIG
\subsection{create\_origin}
\vred
{\bf NAME}
\vred
\begin{itemize}
\vred
\item[] {\bf create\_origin} - convert an event text file into the CSS3.0 {\it .origin} relation.
\vred
\end{itemize}
\vred
{\bf SYNOPSIS}
\vred
\begin{itemize}
\vred
\item[] {\bf create\_origin cmt\_event db\_name}
\end{itemize}
\vred
{\bf DESCRIPTION}
\vred
\begin{quote}
\vred
The shell script {\bf create\_origin} reads from the {\bf cmt\_event} ASCII 
file the first line written in format of input cmt\_event file for 
program {\bf green}, converts the source time and location into a single row 
CSS3.0 relation table, and stores it into {\bf db\_name.origin} file.
The script {\bf create\_origin} only fills out in the {\it .origin} relation 
the following fields: {\it lat, lon, depth, time, orid, jdate, auth}, and
{\it lddate}. The other fields are not important for the {\bf Mineos} package
and they are filled with default values. For more details about the {\it .origin}
relation see below, or the complete description see in (Anderson, J,.et al., 1990).
\vred
\begin{center}
\begin{tabular*}{1.0\textwidth}{@{\extracolsep{\fill}}|lccccl|}  \hline
\multicolumn{2}{|l}{{\it Relation:}} & \multicolumn{4}{l|}{{\bf origin}} \\
\multicolumn{2}{|l}{{\it Description:}} & \multicolumn{4}{l|}{Data on event location and confidence bounds} \\ \cline{1-6}
\multicolumn{1}{|l}{attribute \vspace{-0.07 in}} & \multicolumn{1}{c}{field} & \multicolumn{1}{c}{storage} & \multicolumn{1}{c}{external} & \multicolumn{1}{c}{character} & \multicolumn{1}{l|}{attribute} \\
\multicolumn{1}{|l}{name} &  \multicolumn{1}{c}{no.} & \multicolumn{1}{c}{type} & \multicolumn{1}{c}{format} & \multicolumn{1}{c}{position} & \multicolumn{1}{l|}{description} \\ \hline\hline
lat       &  1 & f4  &  f9.4 & 1-9     & estimated latitude \\
lon       &  2 & f4  &  f9.4 & 11-19   & estimated longitude \\
depth     &  3 & f4  &  f9.4 & 21-29   & estimated depth \\
time      &  4 & f8  &  f17.5& 31-47   & epoch time \\
orid      &  5 & i4  &  i8   & 49-56   & origin id \\
evid      &  6 & i4  &  i8   & 58-65   & event id \\
jdate     &  7 & i4  &  i8   & 67-74   & julian date \\
nass      &  8 & i4  &  i4   & 76-79   & number of associated phases \\
ndef      &  9 & i4  &  i4   & 81-84   & number of lacating phases \\
ndp       & 10 & i4  &  i4   & 86-89   & number of depth phases \\
grn       & 11 & i4  &  i8   & 91-98   & geographic region number \\
srn       & 12 & i4  &  i8   & 100-107 & seismic region number \\
etype     & 13 & c7  &  a7   & 109-115 & event type \\
depdp     & 14 & f4  &  f9.4 & 117-125 & estimated depth from depth phases \\
dtype     & 15 & c1  &  a1   & 127-127 & depth method used \\
mb        & 16 & f4  &  f7.2 & 129-135 & body wave magnitude \\
mbid      & 17 & i4  &  i8   & 137-144 & mb magid \\
ms        & 18 & f4  &  f7.2 & 146-152 & surface wave magnitude \\
msid      & 19 & i4  &  i8   & 154-161 & ms magidd latitude \\
ml        & 20 & f4  &  f7.2 & 163-169 & local magnitude \\
mlid      & 21 & i4  &  i8   & 171-178 & ml magid \\
algorithm & 22 & c15 &  a15  & 180-194 & location algorithm used \\
auth      & 23 & c15 &  a15  & 196-210 & source/originator \\
commid    & 24 & i4  &  i8   & 212-219 & comment id \\
lddate    & 25 & date &  a17 & 221-237 &  load date \\ \cline{1-6}
\end{tabular*}
\end{center}
\vspace{20pt}
\end{quote}
