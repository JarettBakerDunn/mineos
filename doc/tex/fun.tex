\section {fdb and time subroutines/functions}
\subsection {fdb subprograms}
{\bf fdb} (fortran data base) subroutines/functions come from the CU-Boulder {\bf fdb }
FORTRAN 77 library. They
provide a FORTRAN interface to {\it .site, .sitechan, .wfdisc},
and {\it .eigen} relations. Most of {\bf fdb} subroutines/functions have
very short and simple source code. The list of function names with a
brief description are listed below:
\begin{description}
\item{\bf subroutine read\_site}  reads {\it .site} relation from file
\item{\bf subroutine write\_site}  writes {\it .site} relation to file
\item{\bf subroutine default\_site}  sets up default (empty) tuple of 
{\it .site} relation in memory
\item{\bf subroutine read\_sitechan}  reads {\it .sitechan} relation from file
\item{\bf subroutine write\_sitechan}  writes {\it .sitechan} relation to file
\item{\bf subroutine default\_sitechan}  sets up default (empty) tuple of {\it .sitechan} relation in memory
\item{\bf subroutine read\_wfdisc}  reads {\it  .wfdisc} relation from file
\item{\bf subroutine write\_wfdisc}  writes {\it  .wfdisc} relation to file
\item{\bf subroutine put\_wfdisc}  writes for selected {\it .wfdisc} 
tuple associated binary file
\item{\bf subroutine get\_wfdisc}  reads for selected {\it .wfdisc} tuple 
associated binary file
\item{\bf subroutine select\_sitechan} - grouping {\it .sitechan} relation 
by components (single channel or)
\item{\bf subroutine default\_wfdisc}  sets up default (empty) tuple of 
{\it .wfdisc} relation in memory
triple of channels)
\item{\bf subroutine open\_eigen}  opens an {\it .eigen} file and associated 
binary file
\item{\bf subroutine close\_eigen}  closes an {\it .eigen} file and associated 
binary file
\item{\bf subroutine write\_eigen}  writes a single tuple into the 
{\it .eigen} file
\item{\bf subroutine read\_eigen}  reads a single tuple from the 
{\it .eigen} file
\item{\bf subroutine null\_eigen}  sets up default (empty) tuple of 
{\it .eigen} relation in memory
\item{\bf subroutine get\_eigen}  reads current 
{\it .eigen} relation's tuple associated binary file
\item{\bf subroutine put\_eigen}  writes current 
{\it .eigen} relation's tuple associated binary file
\item{\bf integer*4 factor2}  factors an integer number into two factors.
\end{description}
%
%
\subsection {time subprograms}
{\bf time} subroutines/functions perform conversation from human to epoch 
time and vice versa. Function {\bf loctime} returns local system time as 
a text string. A short description is the following:
\begin{description}
\item{\bf subroutine epochtoh(t, year, doy, hour, min, sec)}  converts epoch 
time, {\bf double* t}, into human time , {\bf integer*4 year,doy,hour,min}, 
{\bf real*8 sec}. {\bf doy} is a day starting from the beginning of year
\item{\bf real*8 function  htoepoch (year, doy, hour, min, sec)}  returns 
epoch time. Input arguments are {\bf year, doy, hour}, minutes ({\bf min}), 
{\bf sec}. Here, {\bf integer*4 year,doy,hour,min} and \\
{\bf real*8 sec}
\item{\bf subroutine doytom(year, doy, mon, day)} converts {\bf year}
and  day of year ({\bf doy}) into month ({\bf mon}) and day of 
month({\bf day}). The type of all arguments is {\bf  integer*4}.
\item{\bf integer*4 function mtodoy(year, mon, day)} returns days of year. 
Input arguments are {\bf year}, month ({\bf mon}), and {\bf day}. 
The type of all arguments is {\bf integer*4}.
\item{\bf character*17 function loctime()} returns local system time of form
{\bf mm/dd/yy-hh:mm:ss}
\end{description}
