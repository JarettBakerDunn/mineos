\section{eigcon program}

{\bf eigcon} program makes postprocessing of the {\bf minos\_bran} results and
creates the {\it .eigen} relation table in {\it dbname.eigen},
where {\it dbname} is the database name. 
The program creates a directory called {\it dbname.eigen.dat} in the same directory
where the file {\it dbname.eigen} is located. Under {\it dbname.eigen.dat}
it creates a binary file with a fixed name, {\it eigen}, for storing
the eigenfunctions. This binary file consists of segments with a
fixed length - one segment per eigenfunction.
The {\it eigen} binary file is a portable file. Unlike
the {\bf minos\_bran} binary file it does not have any encapsulation and
can be
easily treated by programs written in various high level languages 
such as C, C++, Perl, etc.
Access to a segment is provided through the {\it .eigen} relation table, namely,
by referencing the path, length,  byte offset, and by type of data. 
The {\it .eigen}
relation table, Section 8, has the structure of an external database file
and can be easily incorporated into relational databases such as
ORACLE, Postgress, MySql, etc.

\subsection {Command line}

\noindent {\bf eigcon}

\noindent {\bf eigcon} $<$ {\it parameter\_file}

\noindent {\bf eigcon} $<<$ WORD \\
\noindent ........  \\
\noindent ........  \\
\noindent WORD

\noindent There are three different ways to start the {\bf eigcon} program:
\begin{itemize}
\item interactive dialog for setting input parameters;
\item single shell command with redirection of the standard input from a parameter
file {\it parameter\_file} containing exactly the same information and
in the same order as in the interactive dialog - one answer per line;
\item direct shell script. In this case, the contents of the parameter file are
directly included in the shell script between delimiters ``WORD". See examples.
\end{itemize}
\noindent The parameter file consists of six lines:
\begin{description}
\item[Line 1:] {\it jcom} \\
{\it jcom} is the type of oscillation.  $jcom=1$ for radial modes, $=2$ for
  toroidal modes, $=3$ for spheroidal modes, and $=4$ for inner core toroidal
  modes. \\
  {\bf Format:} unformatted.
\item[Line 2:] {\it model\_file} \\
  {\it model\_file} is the path to the 1-D input model file. Text string up to
  256 characters long.
\item[Line 3:] {\it max\_depth}\\
  {\it max\_depth} is the depth to cut all output eigenfunctions (in km). All output
  values exists in the interval $[r_n,r_n - max\_depth]$ only. $r_n$ is the radius
  of the  free surface in km. \\
  {\bf Format:} unformatted.
\item[Line 4:] {\it in\_plain\_file} is the path to the input ASCII file. 
  The file is the output model listing  {\it out\_plain\_file} of the
  program {\bf minos\_bran}. See section 3.3.
  Text string up to 256 characters long.
\item[Line 5:] {\it in\_bin\_file} \\
  {\it in\_bin\_file} is the  path to the input FORTRAN binary unformatted file,
which was produced by the program {\bf minos\_bran}.
 See section 1.3.
\item[Line 6:] {\it dbname} \\
  {\it dbname} is the path to the output database name. The path is a string up to 256
  characters long. The path should not
  end with a backslash, ``/". The part of string after the last ``/" or from the
  beginning of the string, if string does not have ``/" at all, is the
  data base name. The data base name must be at least one character long.
\end{description}
%
% Input data
%
\subsection {Input data}

\textbf{\large \emph{model\_file.}} See description in Section 3.2, 
\textbf{\emph{model\_file}}.

\noindent \textbf{\large \emph{in\_plain\_file.}}
This file has been created by {\bf minos\_bran} as an output file.
See description of the file in Section 3.3, \textbf{\emph{out\_plain\_file}}.
Actually, {\bf eigcon} uses only normal mode properties from this file to create
some part of {\bf .eigen} relation table.

\noindent \textbf{\large \emph{in\_bin\_file.}}
This file has been created by {\bf minos\_bran} as an output file.
See description of the file in Section 3.3, \textbf{\emph{out\_bin\_file}}.

\subsection {Output data}

As mentioned above, {\bf eigcon} creates three objects in the file system:
a relational table \\
{\it dbname.eigen}, a directory {\it dbname.eigen.dat}, and a
binary file {\it dbname.eigen.dat/eigen}.

\textbf{\large \emph{dbname.eigen}} file format is described in Section 8 as
the relation table {\it .eigen}. Each line of this file describes the modal
properties of a single eigenfunction and references the binary data
segment in the {\it dbname.eigen.dat/eigen} file.

\textbf{\large \emph{dbname.eigen.dat/eigen}} binary file consists of segments.
Each segment stores normalized eigenfunctions for a single normal mode ($n,l$). 
It consists of words 4 bytes long containing real*4 or integer*4 variables
in the internal computer
format. The field {\it datatype} in {\it .eigen} describes numeric memory
storage data (byte order). There are two common byte orders
for most of computer architectures:
BIG\_ENDIAN (straight order 1-2-3-4) and LOW\_ENDIAN (reverse order 4-3-2-1).
BIG\_ENDIAN is used by SUNS and RISC-oriented platforms, and LOW\_ENDIAN is 
used by PC, VAX and DEC hardware. {\bf eigcon} automaticaly detects the byte order
and places the text strings ``t4" (BIG\_ENDIAN) or ``f4" (LOW\_ENDIAN) into the
{\it datatype} field in the {\it .eigen} relation.

Each segment is logically divided into two parts -  the header and the body. \\
{\bf \emph{The header.}} This part of the segment stores   scalar 
parameters  and normalization coefficients
for the modal properties.
\begin{quote}
\begin{description}
\item[{\bf word 1:} {\tt integer*4}, $n$] - normal mode radial order number $n$. Must
  be equal to {\it norder} in the referencing {\it .eigen} relation.
\item[{\bf word 2:} {\tt integer*4}, $l$] - normal mode angular order 
(harmonic degree) number $l$. Must
  be equal to {\it lorder} in the referencing {\it .eigen} relation.
\item[{\bf word 3:} {\tt real*4}, $\omega$] Normal mode frequency (eigenvalue) in
  physical units rad/s, \mbox{$\;\omega=2\pi/per$}. $per$ is the period field in the
  {\it .eigen} relation.
\item[{\bf word 4:} {\tt real*4}, $q$] - part of the exponential term in the
  attenuation
  expression, \mbox{$e^{-qt}$}, \mbox{$q=0.5*\omega/Q$}, rad/sec.
\item[{\bf word 5:} {\tt real*4}, $r_n$] - the radius normalization coefficient 
  in meters, equal to the radius of the model's free surface.
\item[{\bf word 6:} {\tt real*4}, $v_n$] - the velocity normalization 
  coefficient, $v_n=r_n\;(\pi G\rho_n)^{-1/2}$, where,
$G$ is the gravitational potential constant 
($G=6.6723\cdot 10^{-11}$ $\rm m^3/kg/s^2$),
$\rho_n$ is the density normalization coefficient ($\rho_n=5515$ $\rm kg/m^3$).
The circular frequency normalization coefficient $\omega_n$ is given by
$\omega_n=v_n/r_n$.
\item[{\bf word 7:} {\tt real*4}, $a_n$] - the acceleration normalization 
  coefficient, \mbox{$a_n=10^{20}/(\rho_n r_n^4)$}.
\end{description}
\end{quote}
%
\noindent {\bf \emph{The body - eigenfunction grid.}}
%
\begin{quote}
\begin{description}
\item[{\bf word 8$\;-\;$end:}] {\tt real*4}, $E(nrow,ncol)$ - matrix 
  of $nrow$ rows and $ncol$ columns to store the normalized eigenfunctions. 
  The first column is radius. The other columns are the
  eigenfunctions.
  The number of eigenfunctions depends on modal type. For spheroidal
  modes, it should be 7 columns ($r,\;U,\;U^\prime,\;V,\; V^\prime,\;P,
  \;P^\prime$), for toroidal and others, it should be 3 columns ($r,\;W,\;W^\prime$).
  The matrix is stored in segments by the second index, by columns.
\end{description}
\end{quote}

\noindent Note that {\bf eigcon} performs an additional eigenfunction
normalization. Eigenfunctions for toroidal and the horizontal  
part of spheroidal modes are divided by $(l(l+1))^{1/2}$.
This is done in accordance with theory developed by
Woodhouse \& Dahlen (1978).
The new normalization of eigenfunctions is given by
\[       \omega^2 \int_0^{r_n} \rho(r)\; l(l+1)\;W^{2}(r)r^{2}\;dr = 1 \]
for toroidal modes and
\[       \omega^2 \int_0^{r_n} \rho(r)\; [U^{2}(r)+l(l+1)\;V^{2}(r)]r^{2}\;dr = 1 \]
for spheroidal modes.
%
\subsection{Messages}
Program {\bf eigcon} prints out on the standad output 
device the following messages:

\noindent \textbf {\emph{Dialog messages}}. Copy of input/output dialog
or parameter file in dialog form. For example,

\noindent \texttt {spheroidals (3) or toroidals (2) or radial (1) or  \\
inner core toroidals (4) modes \\
\emph{3} \\
enter name of model file \\
\emph{model\_PREM.txt} \\
enter max depth [km] : \\
\emph{40.} \\
enter name of minos\_bran output text file \\
\emph{PREM\_S} \\
minos\_bran output binary unformatted file name \\
\emph{ePREM\_S} \\
enter path/dbase\_name or dbase\_name to store eigenfunctions: \\
\emph{test\_S} } 

\noindent \textbf {\emph{Info and error messages}}. 

\noindent 1. \texttt{============= Program eigcon ====================} 
\begin{quote}  
Info message. Shows that program {\bf eigcon} starts.
\end{quote}
\noindent 2. \texttt{eigcon: n,nstart,nrad = nnn, mmm, kkk}
\begin{quote}  
Info message. {\tt nnn} is the total number of nodes  of eigenfunction,
{\tt mmm} is the number of staring node after cutting eigenfunction by depth,
and {\tt kkk} is the rest number of nodes after cutting.
\end{quote}
\noindent 3. \texttt{ERR001: eigcon: Input plane and binary files differ: nn, ll, n, l }
\begin{quote}  
\noindent Order numbers of some eigenfunctions {\tt nn} and {\tt ll} in the text 
file differ from the numbers {\tt n, l} in the binary data segment. 
Program is terminated. Check {\it .eigen} relation or create it again.
\end{quote}
\noindent 4. \texttt{ERR002: eigcon: Unknown jcom nnn}
\begin{quote}  
\noindent Error message. The parameter {\tt jcom} with value {\tt nnn} 
is out of range.  Program is terminated. Provide {\tt jcom} with right value.
\end{quote}
\noindent 5. \texttt{ERR003: eigcon: jcom = nnn does not fit mode sss}
\begin{quote}  
\noindent Error message. Impossible combination of jcom and mode. For example,
jcom=2 (toroidal modes) can not used together with {\bf minos\_bran} 
unformatted file for spheroidal modes.
Program is terminated. Provide right jcom and unformatted file.
\end{quote}
\noindent 6. \texttt{ERR004:eigcon: Wrong minos\_bran output text file}
\begin{quote}  
\noindent Error message. Probably the {\bf minos\_bran} plain file was edited.
Never change this file. Program is terminated. Rerun {\bf minos\_bran} again
to create proper plain file.
\end{quote}
