%
% MINOS\_BRAN
%
\section{Examples}

This section contains examples of the interaction with each of the
four programs of the {\bf Mineos} package. Each program can be
run in three different ways: by interactive dialog, redirection from 
an input file, or direct use of a shell script. In each subsection,
examples of running the program in each way are presented. The output
from each approach should be the same. The programs are run in order:
{\bf minos\_bran}, {\bf eigcon}, {\bf green}, {\bf syndat}.

All of the input files named here are included in the standard distribution
of the {\bf Mineos} package.

\subsection{minos\_bran}

In the example given here, {\bf minos\_bran} reads in a model file
called {\it prem\_noocean.txt}, which is a tabular listing of the
PREM model with the ocean filled in with solid crust. The program 
outputs two files: {\it prem\_noocean\_S} and {\it eprem\_noocean\_S}.
The first file contains a listing of some normal mode properties (n, $l$,
frequency, period, phase and group speed, etc) and the second file
contains the eigenfunctions. Both files are needed to be read into 
the eigenfunction renormalization program, {\bf eigcon}. In this example,
the numerical tolerance parameter eps is set to $10^{-10}$ and gravity is taken
into consideration in computing the eigenfunctions only if the frequency
is less than 10 mHz. The  output normal mode properties will be
for spheroidal modes with angular order $l$ values ranging between 1 and 6000, frequencies
ranging between 0 and 166 mHz (i.e., 6 sec), and  radial orders $n$ ranging from 0
to 0 -- that is, only fundamental modes will be computed in this example.

{\bf\emph{Example 1.}} Interactive dialog
\begin{quote}
\texttt{ mathis\% minos\_bran \\ 
input model file: \\
{\bf \emph{prem\_noocean.txt}} \\
output file: \\
{\bf \emph{prem\_noocean\_S}} \\
eigenfunction file (output): \\
{\bf \emph{eprem\_noocean\_S}} \\
enter eps and wgrav \\
{\bf \emph{1e-10 10}} \\
enter jcom (1=rad;2=tor;3=sph;4=ictor) \\
{\bf \emph{3}} \\
enter lmin,lmax,wmin,wmax,nmin,nmax \\
{\bf \emph{1 6000 0.0 166.0 0 0}} \\
mathis\%  }
\end{quote}

\newpage
{\bf\emph{Example 2.}} Redirection of input file \\
Create in the working directory a parameter file named {\tt Param} with
the following contents:
\begin{quote}
\texttt{prem\_noocean.txt \\
prem\_noocean\_S \\
eprem\_noocean\_S \\
1e-10 10 \\
3 \\
1 6000 0.0 166.0 0 0 }
\end{quote}
and start the following command:

\texttt{...\% minos\_bran $<$ Param}


\noindent {\bf\emph{Example 3.}} Direct shell script \\
Include in your sh/csh script the following lines:
\begin{quote}
\texttt{............. \\
minos\_bran $<<$ EOF \\
prem\_noocean.txt \\
prem\_noocean\_S \\
eprem\_noocean\_S \\
1e-10 10 \\
3 \\
1 6000 0.0 166.0 0 0 \\
EOF \\
............. }
\end{quote}

\newpage
%
% EIGCON
%

\subsection{eigcon}

In this example, the two files computed by {\bf minos\_bran} are read in as input,
{\it prem\_noocean\_S} and {\it eprem\_noocean\_S},
together with the input model file {\it prem\_noocean.txt}. {\bf eigcon} renormalizes
the eigenfunctions and outputs them to a depth of 1000 km, in this example. The
renormalized eigenfunctions are placed in an extension of the CSS3.0 data base,
using the relation {\it test\_S.eigen}. This relation points to a file called {\it eigen}
located in a subdirectory called {\it test\_S.dat}. The file {\it eigen} is non-encapsulated,
which allows greater flexibility in access from different platforms and code  from
different compilers. Information about the eigenfunction's byte order is contained
in the {\it .eigen} relation, which is used in subsequent programs to swap bytes appropriately.
Eigenfunctions are computed here to 1000 km for plotting purposes, but for runs
in which earthquakes are no deeper than say 40 km, then 40 would be input here.


{\bf\emph{Example 1.}} Interactive dialog
\begin{quote}
\texttt{mathis\% eigcon \\
spheroidals (3) or toroidals (2) or radial (1) or \\
inner core toroidals (4) modes \\
{\bf \emph{3}} \\
enter name of model file \\
{\bf \emph{prem\_noocean.txt}} \\
enter max depth [km] : \\
{\bf \emph{1000}} \\
enter name of minos\_bran output text file \\
{\bf \emph{prem\_noocean\_S}} \\
minos\_bran output binary unformatted file name \\
{\bf \emph{eprem\_noocean\_S}} \\
enter path\/dbase\_name or dbase\_name to store eigenfunctions: \\
{\bf \emph{test\_S}} \\
mathis\%  }
\end{quote}

{\bf\emph{Example 2.}} Redirection of input file \\
Create in the working directory a parameter file named {\tt Param} with
the following contents:
\begin{quote}
\texttt{3 \\
prem\_noocean.txt \\
1000 \\
prem\_noocean\_S \\
eprem\_noocean\_S \\
test\_S }
\end{quote}
and start the following command:

\texttt{...\% eigcon $<$ Param}

\noindent {\bf\emph{Example 3.}} Direct shell script \\
Include in your sh/csh script the following lines:
\begin{quote}
\texttt{............. \\
eigcon $<<$ EOF \\
3 \\
prem\_noocean.txt \\
1000 \\
prem\_noocean\_S \\
eprem\_noocean\_S \\
test\_S \\
EOF \\
............. }
\end{quote}
\newpage
%
% GREEN
%
\subsection{green}

In this example, there are three principal input files. 

(1) The first input file is the data base name of the {\it .site} and {\it .sitechan}
relations. The entries of the {\it .sitechan} file determine which stations and
channels are used for constructing the Green functions. Channel orientations are
in the {\it .sitechan} file, by station coordinates are in the {\it .site} file.
The data base name for these relations in this example is {\it short}.
There must be, therefore, two pre-constructed files: {\it short.site} and {\it short.sitechan}.

(2) The second input file is the file {\it dblist}.
This file contains the listing of all data base names for the eigenfunction files.
In the previous subsection, the {\it .eigen} relation {\it prem\_noocean\_S.eigen}
was created. So, if that file contains the only normal modes to be used in the
construction of the Green functions, then the file {\it db\_list} would have a 
single entry: prem\_noocean\_S. Note, that only the data base name is included and
not the relation name suffix {\it .eigen}. If other modes are desired, then the file
{\it db\_list} would include the data base names of the other modes. For example,
toroidal modes are usually included in synthetic or SH motions will be ignored. In
this simple example, {\it db\_list} can be considered to have a single entry.

(3) The third input file is the file {\it china\_cmt\_evt}, which is a single-lined listing
containing the coordinates and event parameters of an earthquake in China. The moment tensor
is not used by this program, but by the program {\bf syndat} which follows.

This example will choose modes only between frequencies of 0 and 166 mHz
(i.e., periods greater than 6 sec). It will produce Green functions that are 8000 samples
long. The time sampling specified in the {\it china\_cmt\_evt} file is 1 sec, so this
is a time series length of a little over 2 hours. In many cases, both the minor and
major arc arrivals can be seen.

The program will output a {\it .wfdisc} relation in the data base called {\it green};
that is, a file called {\it green.wfdisc} which points to the waveforms on disk in a default
location.


{\bf\emph{Example 1.}} Interactive dialog
\begin{quote}
\texttt{mathis\% green \\
enter path to db with sta \& stachan: \\
{\bf \emph{short}} \\
enter name of file within list of nmodes db: \\
{\bf \emph{db\_list}} \\
enter input CMT file name: \\
{\bf \emph{china\_cmt\_event}} \\
min and max frequencies to be considered (mHz) : \\
{\bf \emph{ 0 166.}} \\
enter \# pts in greens fns .le.  30000 : \\
{\bf \emph{8000}} \\
enter Green functions output db file name: \\
{\bf \emph{green}} \\
mathis\%  }
\end{quote}
{\bf\emph{Example 2.}} Redirection of input file \\
Create in the working directory a parameter file with the name {\tt Param} with
the following contents:
\begin{quote}
\texttt{short \\
db\_list \\
china\_cmt\_event \\
 0.  166. \\
8000 \\
green }
\end{quote}
and start the following command:

\texttt{...\% green $<$ Param}
\noindent {\bf\emph{Example 3.}} Direct shell script \\
Include in your sh/csh script the following lines:
\begin{quote}
\texttt{............. \\
green $<<$ EOF \\
short \\
db\_list \\
china\_cmt\_event \\
 0.  166. \\
8000 \\
green \\
EOF \\
............. }
\end{quote}
%
% SYNDAT
%
\newpage

\subsection{syndat}

In this example, {\bf syndat} reads in two files. First, there is the event file that
contains the CMT: {\it china\_cmt\_event}. Second, there is the output from the program
{\bf green}: {\it green.wfdisc}. Only the data base name, {\it green}, is input rather
than the whole file name. The output data base name is also specified, which in this
example is {\it Syndat}. The program will output a {\it .wfdisc} relation, {\it Syndat.wfdisc}
in this example, pointing to the waveforms on disk.


{\bf\emph{Example 1.}} Interactive dialog
\begin{quote}
\texttt{mathis\% syndat \\
enter input CMT file name: \\
{\bf \emph{china\_cmt\_event}} \\
enter tensor type: 0 - moment, 1 - nodal plane 1, 2 - nodal plane 2 \\
{\bf \emph{0}} \\
enter input dbname \\
{\bf \emph{green}} \\
enter output dbname \\
{\bf \emph{Syndat}} \\
enter output datatype: 0 -accn, 1 -vel, 2 -displ \\
{\bf \emph{0}} \\
mathis\%  }
\end{quote}
{\bf\emph{Example 2.}} Redirection of input file \\
Create in working directory parameter file with the name {\tt Param} with
the following contents:
\begin{quote}
\texttt{china\_cmt\_event \\
0 \\
green \\
Syndat \\
0}
\end{quote}
and start the following command:

\texttt{...\% syndat $<$ Param}
\newpage
\noindent {\bf\emph{Example 3.}} Direct shell script \\
Include in your sh/csh script the following lines:
\begin{quote}
\texttt{............. \\
syndat $<<$ EOF \\
china\_cmt\_event \\
0 \\
green \\
Syndat \\
0 \\
EOF \\
............. }
\end{quote}
